%% ------------------------------------------------------------------------- %%
\chapter{Introdução}
\label{cap:introducao}

Como já pudemos adiantar no resumo, o presente trabalho de pesquisa científica 
versará sobre a estrutura de dados conhecida como Árvore de Segmentos.

Tal estrutura é comumente utilizada em programação competitiva por tratar-se de 
uma ferramenta versátil e extremamente importante na solução de determinados 
tipos de problema, sendo os mais comuns aqueles em que se deseja perguntar qual o 
elemento mínimo de um intervalo inúmeras vezes dado uma determinada sequência de dados.

No entanto, a pesquisa a que se propõe vai além de apenas demonstrar conceitos 
acerca da ferramenta mencionada acima.

Tendo em vista a escassez de material em língua portuguesa e a grande quantidade 
de material que demonstra apenas uma implementação “caixa preta”, esta pesquisa 
pretende solidificar as formas e as possibilidades de utilização da ferramenta, bem como 
a problemática envolvida em sua escolha, as formas de implementação e, ainda, 
corroborar sua utilização através da solução de problemas práticos e 
ordinariamente enfrentados na esfera da programação competitiva.

Além disso, este estudo aprofunda-se, inclusive, em uma das formas de adaptação 
de implementação da árvores de segmentos, a propagação "preguiçosa", demonstrando o 
aumento de possibilidades de aplicação da ferramenta.



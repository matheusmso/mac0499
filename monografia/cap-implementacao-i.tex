%% ------------------------------------------------------------------------- %%
\chapter{A árvore de segmentos}
\label{cap:implementacao-i}

Apresentada a problemática em que se é possível utilizar a estrutura em análise, 
passamos a especificar suas características.

A árvore de segmentos é um estrutura de dados baseada no paradigma de divisão e 
conquista e pode ser pensada como uma árvore de intervalos avaliados em um vetor 
fundamental construído de forma a possibilitar perguntas sobre intervalos do 
vetor e modificações sobre estes mesmos intervalos de forma extremamente 
eficientes.

A estrutura da árvore de segmentos se baseia nos seguintes fatos:
\begin{itemize}
    \item é uma árvore binária, ou seja, cada nó possui dois filhos;
    \item cada nó representa um intervalo do vetor fundamental e este nó pode 
    conter informações sobre mais do que uma função aplicada em intervalos do 
    vetor fundamental;
    \item cada folha da árvore está associada a um intervalo que contém apenas 
    um elemento;
    \item subindo na árvore, cada pai representa a união dos resultados das 
    funções aplicadas nos intervalos (disjuntos) representados por seus dois 
    filhos;
    \item o nó raiz representa o vetor inteiro.
\end{itemize}

Para criar tal estrutura precisamos de três operações básicas, quais sejam:

\section{Construção}

Para construir uma árvore de segmentos precisamos inicializar valores nos nós 
desta, valores estes que representam o vetor fundamental. É possível criar a 
árvore de segmentos tanto de forma top-down (recursiva) ou bottom-up 
(iterativa). A construção top-down popula o valor do nó raiz que, por sua vez, 
implica em chamadas recursivas para cada uma das metades do intervalo que contém 
o intervalo em questão, resultando em outras chamadas para as respectivas metades; 
os casos base são as folhas, que podem ser calculados imediatamente dos valores 
originais do vetor. Uma vez calculados os valores das duas metades do intervalo, 
calcular o valor do pai se resume apenas em aplicar as funções nos respectivos 
resultados dos subintervalos.

\section{Atualização}

Para atualizar uma árvore de segmentos precisamos modificar o valor de algum 
elemento do vetor fundamental. Para isso modificamos o valor da folha referente 
a este elemento. As outras folhas não são afetadas por essa atualização uma vez 
que cada folha está associada a apenas um elemento do vetor. Uma vez modificada 
a folha em questão o nó pai desta folha é afetado e pode ter seu valor modificado, 
assim como o nó avô e assim por diante até a raiz, mas nenhum outro nó é afetado. 
Novamente para executar uma atualização podemos optar pelos métodos top-down 
(recursivo) ou bottom-up (iterativo). Começamos fazendo a chamada na raiz, o 
que gera chamadas recursivas para apenas um dos filhos, aquele cujo intervalo 
contém o elemento que desejamos atualizar, o outro nó não é afetado. O caso base 
novamente é a folha associada ao elemento do vetor que desejamos atualizar. Depois 
de finalizada a recursão basta avaliar novamente o resultado da função dos filhos 
depois da atualização de um deles.

\section{Pergunta}

Para perguntar à uma árvore de segmentos, precisamos determinar o resultado de 
uma função aplicada a um determinado intervalo do vetor fundamental. A execução
da pergunta é bastante complexa e acompanhará o exemplo. Assim, retomamos o 
exemplo do capítulo anterior:

Dado o vetor:

$$(1, 8, 2, 5, 4, 11, 3, 7)$$

queremos perguntar o menor valor entre o segundo e o quinto, incluso. 

Representaremos essa pergunta da seguinte forma: $f(2,5)$. Cada nó da árvore de 
segmentos contém o mínimo de algum intervalo, a raiz contém $f(1, 8)$ que é o 
menor valor do vetor inteiro, seu filho esquerdo possui $f(1, 4)$ e seu filho 
direito $f(5, 8)$. Podemos perceber que $f(2, 5) = min(f(2, 2), f(3, 4), f(5, 5))$.

Conforme explanado anteriormente, são diversas as formas de utilização da árvore de 
segmentos, sendo natural que, no decorrer das etapas e treinamentos de programação 
competitiva, sua utilização se torne intuitiva para os casos aplicáveis.

A seguir, veremos como aplicá-la de forma prática.



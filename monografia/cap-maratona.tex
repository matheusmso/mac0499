%% ------------------------------------------------------------------------- %%
\chapter{Problemas Selecionados}
\label{cap:maratona}

Neste item dissertaremos acerca de vários exemplos de problemas de programação competitiva
cuja solução envolve a utilização da estrutura de dados árvore de segmentos. 
Nos primeiros problemas encontraremos a solução com a
aplicação direta da implementação da árvore de segmentos, não sendo necessário comentários adicionais, haja vista a vasta explanação anterior.
Através da solução destes problemas, será possível que os estudantes de programação competitiva verifiquem seu entendimento sobre o tema
e se suas implementações dos algoritmos estão corretas. Os  problemas que se seguirão versarão sobre a implementação da árvore de segmentos com a propagação preguiçosa.
Todos os problemas
estão listados em ordem de dificuldade e possuem um link para um juiz online com o respectivo problema,
para que o estudante interessado possa submeter e verificar sua própria solução. Além disso, uma implementação
em C++ da solução de cada problema está disponível em \url{https://linux.ime.usp.br/~matheusmso/mac0499/codigos.html}.

\section{RPLN - Negative Score}
Este exemplo demonstra a aplicação direta da estrutura estudada neste trabalho, mais precisamente a detalhada no capítulo $4$. Importante notar que esse problema não requer implementação da funcionalidade de atualização.
\subsection{URL}
\url{https://www.spoj.com/problems/RPLN/}
\subsection{Descrição da entrada}
A primeira linha dos dados de teste começará com um inteiro T representando o número de  casos, cada um dos casos começa com dois inteiros N e Q. N inteiros seguirão indicando a pontuação em cada avaliação, depois disso, Q consultas serão iniciadas, cada consulta consistirá em dois inteiros A e B, o intervalo de interesse.
\subsection{Descrição da saída}
Você deve produzir a linha que indica qual teste estamos imprimindo e, em seguida, o resultado de cada consulta, lembre-se, que estamos em busca da pior pontuação da avaliação A à avaliação B, inclusive.
\subsection{Exemplo}
\textbf{Entrada:} \\
2 \\ 
5 3 \\
1 2 3 4 5\\
1 5\\
1 3\\
2 4\\
5 3\\
1 -2 -4 3 -5\\
1 5\\
1 3\\
2 4\\

\textbf{Saída:} \\
\textbf{Scenario \#1:} \\
1\\
1\\
2\\
\textbf{Scenario \#2:} \\
-5\\
-4\\
-4\\
\subsection{Solução}
Basta utilizar a estrutura estudada, em sua forma minimal, para responder as perguntas de mínimo em um dado intervalo.


\section{INVCNT - Inversion Count}
\subsection{URL}
\url{https://www.spoj.com/problems/INVCNT/}
\subsection{Descrição da entrada}
A primeira linha contém T, o número de casos de teste seguido por um espaço em branco. Cada um dos testes começa com um número $N$. Cada uma das linhas que seguem contém um dos elementos do vetor que gostaríamos de contar o número de inversões.
\subsection{Descrição da saída}
Para cada saída de teste, uma linha dando o número de inversões de A.
\subsection{Exemplo}
\textbf{Entrada:} \\
2\\
\\
3\\
3\\
1\\
2\\
\\
5\\
2\\
3\\
8\\
6\\
1\\

\textbf{Saída:} \\
2\\
5\\
\subsection{Solução}
Neste problema gostaríamos de responder a seguinte pergunta: dado um número $x$ quantos números maiores que $x$ já estão presentes? Como os números estão limitados a $10^7$ podemos alocar um vetor booleano que marcará se um número está ou não presente na lista. Usando essa estrutura, resolver o problema se resume a processar os números em ordem e perguntar para cada um quantos maiores que ele existem, somar isso a resposta e depois adicioná-lo à estrutura.

Para atingir esse resultado basta alterarmos um pouco a função de atualização:
\begin{lstlisting}
void update(int pos, int value, int node = 1, int l = 0, int r = N) {
  if (l + 1 == r) {
    segtree[node]++;
    return;
  }
  int mid = (l + r)/2;
  if (pos < mid)
    update(pos, value, 2*node, l, mid);
  else
    update(pos, value, 2*node+1, mid, r);
  segtree[node] = join(segtree[2*node], segtree[2*node+1]);
}
\end{lstlisting}
Importante notar que para esse caso, dado que não existe uma lista de origem, não precisamos construir a árvore, ou seja, não precisamos implementar a função $build$.

\section{Potentiometers}
\subsection{URL}
\url{https://icpcarchive.ecs.baylor.edu/index.php?option=com_onlinejudge&Itemid=8&page=show_problem&problem=192}

\subsection{Descrição da entrada}
Cada caso começa com $N$, o número de potenciômetros no array. Cada uma das próximas $N$ linhas contém um número entre $0$ e $1000$, as resistências iniciais dos aparelhos na ordem $1$ a $N$. Cada uma das linhas contém uma ação. Existem três tipos de ação:
\begin{itemize}
    \item "S $x$ $r$" - define o aparelho $x$ para $r$ Ohms. $x$ é um número válido de potenciômetro e r é entre $0$ e $1000$.
    \item "M $x$ $y$" - meça a resistência entre o terminal esquerdo do potenciômetro $x$ e o terminal direito de potenciômetro $y$. Ambos os números serão válidos e $x$ é menor que ou igual a $y$.
    \item "END" - final deste caso. Aparece apenas uma vez no final de uma lista de ações.
\end{itemize}

$N = 0$ indica o fim dos casos de teste.
\subsection{Descrição da saída}
Para cada caso na entrada, produza uma linha que indique o caso de teste em questão.
Para cada medição na entrada, imprima uma linha contendo um número: a resistência medida
em Ohms. As ações devem ser aplicadas ao array de potenciômetro na ordem dada na entrada.
\subsection{Exemplo}
\textbf{Entrada:} \\
3 \\
100 \\
100 \\
100 \\
M 1 1 \\
M 1 3 \\
S 2 200 \\
M 1 2 \\ 
S 3 0 \\
M 2 3 \\
END \\
10 \\
1 \\
2 \\
3 \\
4 \\
5 \\
6 \\
7 \\
8 \\
9 \\
10 \\
M 1 10 \\
END \\
0 \\

\textbf{Saída:} \\
Case 1: \\
100 \\
300 \\
300 \\
200 \\
Case 2: \\
55 \\

\subsection{Solução}
Esse problema também é bastante direto, precisamos atualizar elementos da lista e responder para perguntas do tipo qual a soma deste sub intervalo.

Para isso basta utilizar a estrutura detalhada no capítulo $4$ com a função de união de dois nós apresentada na seção $5.5$.

\section{Banco do Faraó}
\subsection{URL}
\url{https://br.spoj.com/problems/BANFARAO/}

\subsection{Descrição da entrada}
A entrada é composta por diversas instâncias. A primeira linha da entrada contém um inteiro $T$ indicando o número de instâncias.
A primeira linha de cada instância contém um inteiro $N$, indicando o número de contas no Banco do Faraó. A segunda linha de cada instância contém $N$ inteiros, entre $-10000$ até $10000$, indicando os saldos nas contas dos correntistas. A terceira linha contém um inteiro $Q$ indicando o número de consultas que serão feitas. Cada uma das $Q$ linhas seguintes contém dois inteiros $A$ e $B$ indicando o intervalo que deve ser consultado.

\subsection{Descrição da saída}
Para cada instância seu programa deve produzir $Q$ linhas na saída, sendo uma para cada consulta. Cada uma dessas linhas deve conter dois inteiros: o primeiro representa a soma do intervalo com maior soma, e o segundo, o número de elementos desse intervalo. Caso haja mais de um intervalo com maior soma, imprima o número de elementos naquele com maior número de elementos.

\subsection{Exemplo}
\textbf{Entrada:} \\
3 \\
3 \\
-1 -2 -3 \\
1 \\
1 1 \\
8 \\
1 2 -1 4 9 8 -1 2 \\
4 \\
1 3 \\
1 4 \\
2 5 \\
7 8 \\
3 \\
0 0 0 \\
1 \\
1 3 \\

\textbf{Saída:} \\
-1 1 \\
3 2 \\
6 4 \\
14 4 \\
2 1 \\
0 3 \\

\subsection{Solução}
Para responder qual o subintervalo de maior soma e quantos elementos ele possúi, precisamos guardar para cada intervalo do vetor as seguintes informações:
\begin{itemize}
    \item a soma do subintervalo com maior soma do intervalo e seu tamanho.
    \item a maior soma de prefixo do intervalo e seu tamanho.
    \item a maior soma de sufixo do intervalo e seu tamanho.
    \item a soma total do intervalo e seu tamanho.
\end{itemize}

Para isso definimos o nó da árvore da seguinte forma:
\begin{lstlisting}
struct node {
  pair<int, int> bst;
  pair<int, int> pre;
  pair<int, int> suf;
  pair<int, int> tot;
};
node seg[4*N];
\end{lstlisting}

Dessa forma podemos usar a estrutura definida no capítulo $4$ e basta adaptar a função que une dois nós. Dados dois nós para uni-los basta fazer algumas comparações:
\begin{itemize}
    \item a soma do subintervalo com maior soma desse novo intervalo é o máximo entre $3$ fatores, o subintervalo de maior valor do nó esquerdo, o subintervalo de maior valor do nó direito e a soma do maior sufixo do nó esquerdo com o maior prefixo do nó direito.
    \item a maior soma de prefixo desse novo intervalo é o máximo entre a maior soma de prefixo do nó esquerdo e o total do nó esquerdo somado com a maior soma de prefixo do nó direito.
    \item a maior soma de sufixo desse novo intervalo é o máximo entre a maior soma de sufixo do nó direito e o total do nó direito somado com a maior soma de sufixo do nó esquerdo.\
    \item a soma total desse novo intervalo é simplesmente a soma dos totais dos nós esquerdo e direito.
\end{itemize}

\begin{lstlisting}
node join(node a, node b) {
  if (a.bst.second == 0) return b;
  if (b.bst.second == 0) return a;
  node aux;
  aux.tot = a.tot+b.tot;
  aux.bst = max({a.bst, b.bst, a.suf+b.pre});
  aux.pre = max(a.pre, a.tot+b.pre);
  aux.suf = max(b.suf, b.tot+a.suf);
  return aux;
}
\end{lstlisting}
As duas primeira linhas da função tratam os casos extremos onde um dos nós representa um intervalo vazio.

E a criação de um novo nó segue a seguinte prática uma vez que um elemento é o total, melhor prefixo sufixo e soma.
\begin{lstlisting}
void build(int no = 1, int l = 0, int r = n) {
  if (l + 1 == r) {
    seg[no].bst = make_pair(v[l], 1);
    seg[no].pre = make_pair(v[l], 1);
    seg[no].suf = make_pair(v[l], 1);
    seg[no].tot = make_pair(v[l], 1);
    return;
  }
  int mid = (l + r)/2;
  build(2*no, l, mid);
  build(2*no+1, mid, r);
  seg[no] = join(seg[2*no], seg[2*no+1]);
}
\end{lstlisting}

Esse problema não necessita a implementação da função de atualização. O problema GSS3 é bastatnte similar porém exige essa funcionalidade.

\section{Interval Product}
\subsection{URL}
\url{https://uva.onlinejudge.org/index.php?option=onlinejudge&page=show_problem&problem=3977}

\subsection{Descrição da entrada}
A primeira linha contém dois inteiros $N$ e $k$, indicando respectivamente o número de elementos na sequência e o número de rodadas do jogo. A segunda linha contém $N$ inteiros $x_i$ que representam os valores iniciais da sequência. Cada uma das próximas $k$ linhas descreve um comando e começa com uma letra maiúscula que é "C" ou "P". Se a letra for 'C', a linha descreve um comando de mudança e a letra é seguida por dois inteiros $i$ e $v$ indicando que $x_i$ deve receber o valor $v$. Se a letra for "P", a linha descreve um comando do produto e a letra é seguida por dois inteiros $i$ e $j$ indicando que o produto de $x_i$ a $x_j$, inclusive, deve ser calculado. Dentro de cada caso de teste, há pelo menos um comando do produto.
\subsection{Descrição da saída}
Para cada caso de teste, imprima uma linha com uma cadeia representando o resultado de todos os comandos do produto no caso de teste. O $i$-ésimo caractere da string representa o resultado do i-ésimo comando do produto. Se o resultado do comando for positivo, o caractere deve ser "+" (mais); se o resultado for negativo, o caractere deve ser "-" (menos); se o resultado for zero, o caractere deve ser "0" (zero).
\subsection{Exemplo}
\textbf{Entrada:} \\
4 6 \\
-2 6 0 -1 \\
C 1 10 \\
P 1 4 \\
C 3 7 \\ 
P 2 2 \\
C 4 -5 \\
P 1 4 \\
5 9 \\
1 5 -2 4 3 \\
P 1 2 \\
P15 \\
C 4 -5 \\
P 1 5 \\ 
P 4 5 \\
C 3 0 \\ 
P 1 5 \\
C 4 -5 \\
C 4 -5 \\
\textbf{Saída:} \\
0+- \\
+-+-0 \\

\subsection{Solução}
Para responder se o resultado do produto de um intervalo é positivo negativo ou zero basta que guardemos duas informações:
\begin{itemize}
    \item quantos números negativos existem num dado intervalo.
    \item quantos zeros existem num dado intervalo.
\end{itemize}
Assim basta usarmos duas árvores se segmento, muito parecidas com a utilizada no problema $6.2$, ou seja, uma árvore de segmento que guarda qutos elementos existem num dado intervalo, uma árvore de segmento de soma onde o vetor de origem é $1$ se o elemento está presente e $0$ caso contrário.

Para isso alteramos a função de construção para representar essa lógica:
\begin{lstlisting}
void build(int node = 1, int l = 0, int r = n) {
  if (l + 1 == r) {
    if (v[l] == 0)
      segz[node] = 1;
    else if (v[l] < 0)
      segn[node] = 1;
    else
      segz[node] = segn[node] = 0;
    return;
  }
  int mid = (l + r)/2;
  build(2*node, l, mid);
  build(2*node+1, mid, r);
  segn[node] = join(segn[2*node], segn[2*node+1]);
  segz[node] = join(segz[2*node], segz[2*node+1]);
}
\end{lstlisting}
onde $segz$ é a árvore que representa o vetor de presença de zeros e $segn$ a árvore que representa a presença de negativos.

A atualização também precisa refletir essa mudança:
\begin{lstlisting}
void update(int pos, int value, int node = 1, int l = 0, int r = n) {
  if (value > 0) {
    segn[node] -= v[pos] < 0;
    segz[node] -= v[pos] == 0;
  }
  else if (value < 0) {
    segn[node] += v[pos] >= 0;
    segz[node] -= v[pos] == 0;
  }
  else {
    segn[node] -= v[pos] < 0;
    segz[node] += v[pos] != 0;
  }
  if (l + 1 == r) {
    v[pos] = value;
    return;
  }
  int mid = (l + r)/2;
  if (pos < mid)
    update(pos, value, 2*node, l, mid);
  else
    update(pos, value, 2*node+1, mid, r);
  segz[node] = join(segz[2*node], segz[2*node+1]);
  segn[node] = join(segn[2*node], segn[2*node+1]);
}
\end{lstlisting}

\section{GSS3 - Can you answer these queries III}
\subsection{URL}
\url{https://www.spoj.com/problems/GSS3/}

\subsection{Descrição da entrada}
A primeira linha de entrada contém um inteiro $N$. A linha a seguir contém $N$ inteiros, representando a sequencia $a_1$..$a_N$.
A terceira linha contém um inteiro $M$. As próximas $M$ linhas contêm as operações no seguinte formato:
\begin{itemize}
    \item 0 $x$ $y$: modifica $a_x$ para $y$.
    \item 1 $x$ $y$: impressão $max{[a_i + a_{i + 1} + .. + a_j | x \leq i \leq j \leq y}$.
\end{itemize}

\subsection{Descrição da saída}
Para cada consulta, imprima um inteiro como o problema requerido.
\subsection{Exemplo}
\textbf{Entrada:} \\
4 \\
1 2 3 4 \\
4 \\
1 1 3 \\
0 3 -3 \\
1 2 4 \\
1 3 3 \\

\textbf{Saída:} \\
6 \\
4 \\
-3 \\

\subsection{Solução}
Como citado no problema $6.4$ esse problema é basicamente uma extensão dele. Precisamos apenas implementar a funcionalidade de atualização. Vale notar que não precisamos mais do tamanho do intervalo o que simplifica um pouco a estrutura:
\begin{lstlisting}
void update(int pos, lint value, int no = 1, int l = 0, int r = n) {
  if (l + 1 == r) {
    segtree[no].bst = segtree[no].pre = segtree[no].suf = segtree[no].tot = v[pos] = value;
    return;
  }
  int mid = (l + r)/2;
  if (pos < mid)
    update(pos, value, 2*no, l, mid);
  else
    update(pos, value, 2*no+1, mid, r);
  segtree[no] = join(segtree[2*no], segtree[2*no+1]);
}
\end{lstlisting}
Sempre que atualizarmos um elemento do vetor, esse elemento é representado por um nó e esse nó tem como total, melhor soma, melhor prefixo e melhor sufixo exatamente o seu valor na posição referente no vetor.

\section{HORRIBLE - Horrible Queries}
\subsection{URL}
\url{https://www.spoj.com/problems/HORRIBLE/}

\subsection{Descrição da entrada}
Na primeira linha você receberá $T$, número de casos de teste.
Cada caso de teste começará com $N$ e $C$. Depois disso, você receberá $C$:
\begin{itemize}
    \item 0 $p$ $q$ $v$ - adicionar $v$ para todos os elementos entre $p$ e $q$ inclusive.
    \item 1 $p$ $q$ - imprimir a soma de todos os elementos entre $p$ e $q$ inclusive.
\end{itemize}
\subsection{Descrição da saída}
Imprima as respostas das consultas.
\subsection{Exemplo}
Entrada:\\
1\\
8 6\\
0 2 4 26\\
0 4 8 80\\
0 4 5 20\\
1 8 8 \\
0 5 7 14\\
1 4 8\\

Saída:\\
80\\
508\\

\subsection{Solução}

Aplicação direta da estrutura apresentada no capítulo $5$.
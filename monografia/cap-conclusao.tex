%% ------------------------------------------------------------------------- %%
\chapter{Conclusão}
\label{cap:conclusao}

O desenvolvimento da presente pesquisa possibilitou demonstrar a capacidade e a importância da utilização da estrutura de dados chamada de árvores de segmentos no ambiente de programação competitiva.

De um modo geral, os alunos dedicados à este esporte não têm à sua inteira disposição um vasto e claro aparato acerca do tema, o que motivou a elaboração deste material.

Ao esclarecer com minúcias as possibilidades de utilização da ferramenta, bem como uma das formas de adaptação de implementação da árvores de segmentos, a propagação "preguiçosa", demonstrando o aumento de possibilidades de aplicação da ferramenta e demonstrar os casos práticos de aplicação, foi possível elucidar ao estudante e participante de programação competitiva, quando e como deverá e poderá utilizar a árvores de segmentos na resolução dos problemas.
 
Tendo em vista a importância do assunto, o desenvolvimento do presente trabalho se deu de maneira desafiadora, diante da necessidade de traduzir o material internacional e o pouco material nacional de uma maneira didática para ser aproveitado aos promissores competidores das maratonas de programação.

Nesse sentido,concluímos a presente análise para que possa ser aproveitada como grande ajuda para os participantes de programação competitiva interessados em conhecer e perscrutar o tema aqui abordado.



